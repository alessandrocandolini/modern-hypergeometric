
%*******************************************************
% Chapter 1
%*******************************************************

%\myChapter[Fuchsian differential equations]{Fuchsian 
%	differential equations of the second order in the complex domain}


\myChapter{Fuchsian differential equations}
\label{chap:fuchs}


%\begin{approfondimento}
%Integrals  of Fuchsian differential equations are a powerful source of higher trascendental
%functions.
%\begin{flushright}
%	--- Folklore
%\end{flushright}
%\end{approfondimento}

%In this chapter, we discuss a number of facts
This chapter is mainly concerned with existence and properties (analytic structure, many-valuedness,
convergent power series representations,%
\footnote{Asymptotic (divergent) power series
   representations, integral representations or other kind of series
   representations are not discussed in general.}
etc) of the integrals of linear homogeneus ordinary differential equations of the
second order in which the independent variable is complex and whose coefficients
are single-valued and analytic complex functions of the independent variable, having
at most finitely many isolated singular points in (a given domain of) the
complex plane.



%discuss 
%Fuchsian theory of linear
%homogeneus ordinary differential equation (of the second order) in the complex domain.
%First, we shall 


%More precisely, we shall consider equations of the solutions of second-order
%whose  coefficients will be assumed to be single-valued analytic complex-valued
%functions having at most a finite number of isolated singularities in their
%domain of definition.

%\footnote{Not all special functions are constructed from solutions of
%\emph{linear} differential equations. Some are constructed from solutions of
%certain \emph{non-linear} differential equations (\eg, the non-linear
%Painlev\'e trascendents), others, like the Euler's gamma function, does not
%appear to satisfy any simple ordinary differential equation (for sure, the
%gamma function cannot satisfy any algebraic differential equation whose
%coefficients are rational functions, this result is known as H\"older
%theorem).}

Several, through  not all,%
\footnote{Riemann's zeta function and Euler's gamma function are examples of
   functions which are essentially characterized by \emph{functional} equations
   instead of differential equation.  (For instance, H\"older's theorem states that
   the gamma function does not satisfy \emph{any} algebraic differential equation
   whose coefficients are rational functions.) Other higher trascendental
   functions, like the Painlev\'e trascendents, arise in connection with the
   solution of certain \emph{non}-linear ordinary differential equations in the
   complex field (see, \eg, \textcite{Ince:1956}).}
   higher trascendental functions of applied
mathematics satisfy differential equations of this form,
   and the results of this chapter
provide a useful  tool for establishing in a systematic way some of their
properties and representations, working directly from the differential
equation satisfied by them.% 
\footnote{Someone might prefer to begin
   with integral or series representations, 
   and to write down the differential
   equations only later.} In particular, we shall put these
results at work in the following chapter discussing Gauss' hypergeometric
differential equation and its solutions.

This chapter is mostly based on \textcite[\S~V]{Tricomi:1961};
\textcite[\S~2]{Wang.Guo:1989}. See also \textcite[\S~V]{Smirnov:1964};
\textcite[\S~X]{Whittaker.Watson:1927}.  In what follows, knowledge of
the theory of analytic  functions of one complex variable is assumed.%
\footnote{%
   There are many books on the theory of analytic functions of one complex
   variable.  Among my favourite ones are the monographies by
   \textcite{Ablowitz.Fokas:2003,Stein.Shakarchi:2003,Marsden.Hoffman:1987,Greene.Krantz:2006};
   or consult any good book on complex analysis.}


%\section{General theory of linear differential equations in the complex field}

\section[Solution in the neighbourhood of an ordinary point]{Existence and
   uniqueness of the solution in the neighbourhood of an ordinary
   point}\label{sec:existence}

%\section{Local solution about an ordinary point}
%\label{sec:existence}
%\section{Ordinary and singular points}

A full and systematic account of the general theory of ordinary differential
equations or systems of ordinary differential equations in the complex field,
although useful in practise, is beyond the scope of this handout. For a broader
discussion, see, \eg, the previously mentioned book by \textcite{Tricomi:1961}
or \textcite{Ince:1956}.

%\nocite{Ilyashenko.Yakovenko:2008} 
%\nocite{Teschl:2011}

We \graffito{Second-order homogeneus linear ordinary differential equations in
   the complex field\ldots} shall restrict our attention to a linear homogeneus
ordinary differential equation of the second order,% 
\footnote{%
   The result of the first sections of this chapter apply with obvious
   generalizations to the case of linear equations of arbitrary order, but
   the second order is enough to our purposes and leads to easier
   notations.
}
which can always be written in such a way that the coefficient of the
second-order derivative is identically equal to one (this will be taken to be
the standard form of the equation of the equation):
\begin{dmath}[label={ode}, frame]
   u'' (z) + p(z) u'(z) + q(z) u(z) = 0 ,
\end{dmath}
where $u(z)$ is the unknown (complex-valued) function and $z$ denotes the
independent complex variable. The coefficients $p(z)$ and $q(z)$ are assigned
complex-valued functions defined on some given set of the complex $z$-plane.
Primes denote differentiation, in the complex sense,
with respect to the variable $z$.

Throughout this chapter, 
\graffito{\ldots whose coefficients are single-valued and analytic up to
   finitely many isolated singularities}
%in a given domain of the complex plane.}
we shall assume that  each of the coefficients $p(z)$ and
$q(z)$ is a
\emph{single-valued}
function, defined and \emph{analytic} on some given \emph{domain}%
\footnote{\ie, an open and connected subset of $\C$.}
$\Omega
\subset\C$ 
of the complex $z$-plane (eventually, $\Omega = \C$), with the possible exception of \emph{at most} a finite number of
\emph{isolated} singular points%
\footnote{which may be either poles or essential singularities of the coefficients.}
within $\Omega$.


%\begin{mdframed}[backgroundcolor=lightgray,
%roundcorner=5pt, 
%skipabove=10pt,skipbelow=10pt]

%Let us briefly summarize  the main definitions concerning the theory of analytic
%functions of one complex variable.
%There are many books on the subject, 
%among my favourite ones are the monographies by
%\textcite{Ablowitz.Fokas:2003,Stein.Shakarchi:2003,Marsden.Hoffman:1987,Greene.Krantz:2006}.
%%Hille:1959
%See also \textcite[\S~1]{Smirnov:1964} and
%\textcite[\S~4]{Giaquinta.Modica:2007}, \textcite[\S~4]{Godement:2004},
%or consult any other book on complex analysis.
%%\footcite{Hille:1959, Ablowitz.Fokas:2003,Stein.Shakarchi:2003,
%%Marsden.Hoffman:1987, Smirnov:1964, Giaquinta.Modica:2007, Godement:2004}


%Let $\Omega$ be an \emph{open} set in $\C$, $z_{0}\in\C$ any point of
%\graffito{Brief resume on what is an holomorphic function of one complex variable}
%$\Omega$ and $f:\Omega\rightarrow\C$ a complex-valued function defined on $\Omega$.
%$f$ is said to be ``differentiable'' in the complex sense at $z_{0}$  if the
%following limit exists:
%\begin{dmath*}
%\lim_{z\rightarrow z_{0}} \frac{f(z) -f(z_{0})}{z-z_{0}} .
%\end{dmath*}
%(Remember that if the limit exists then it must be equal along any restriction.)
%If this is the case, such limit will be denoted by $f'(z_{0})$ and will be
%called the ``complex derivative'' of $f$ at $z_{0}$.  If $f$ admits complex
%derivative at every point of $\Omega$, then it is said to be ``holomorphic'' in
%$\Omega$.  This condition has far-reaching conseguences, in particular it
%implies  (Taylor's theorem) that $f(z)$ can be expanded in a convergent power
%series near $z_{0}$ having non-zero radius of convergence (\ie, $f(z)$ is
%\emph{analytic} on $\Omega$). Since the converse is also true, \ie, any
%convergent power series defines a holomorphic function in its disk of
%convergence, then  it is possible to identify analytic and holomorphic functions
%on $\Omega$.  Sometimes, we shall say that $f$ is analytic at $z_{0}$: This is a
%shortcut to say that there exists a neighbourhood of $z_{0}$ on which $f$ is
%analytic.

%We look for solutions of \eqname s~\eqref{eq:ode} satisfying 

%\subsection{Classification of ordinary and singular points}

Let us classify the point of $\Omega$ according to whether or not they are
isolated singular points of the coefficients of the differential equation.%
\footnote{%
   This classification holds for points $z_{0}$ at \emph{finite} distance
   in the complex plane. The classification of the point at infinity (in
   the case $\Omega = \C$) requires special treatment and will be discussed
   separately in \S~\ref{sec:point at infinity}.
}

\begin{definition}

   Let $z_{0}\in\Omega$.  \graffito{Definition of ordinary point of the
      differential equation} $z_{0}$ is called an ``ordinary'' point of
   Eq.~\eqref{eq:ode} if the coefficient functions $p(z)$ and $q(z)$ are
   both analytic at $z_{0}$.%
   \footnote{%
      As usual, we shall see that a function is analytic \emph{at} a
      point $z_{0}$ to say that the function is analytic in an (open)
      neighbourhood of $z_{0}$.
   } 
   Otherwise, if $z_{0}$ is a singular point of either $p(z)$ or $q(z) $ or
   both, then $z_{0}$ is said to be a ``singular'' point of
   Eq.~\eqref{eq:ode}.

\end{definition}

Let $z_{0}\in\Omega$ an ordinary point of Eq.~\eqref{eq:ode}.  We look for
solutions $u(z)$ \graffito{Initial conditions} of Eq.~\eqref{eq:ode} defined in
a suitable neighbourhood (to be determined) of $z_{0}$ and which satisfy the
following initial conditions at $z_{0}$:
\begin{dmath}[label={ic}]
   \left\lbrace
   \addtolength{\arraycolsep}{-3pt}
   \begin{array}{rcl}
      u(z_{0}) &=& \alpha \\
      u'(z_{0})&=&\beta
   \end{array}
   \right.
\end{dmath}
where $\alpha$ and $\beta$ are arbitrary-assigned complex numbers.
Eq.~\eqref{eq:ode} together with the initial conditions~\eqref{eq:ic} constitute
what is called an ``initial valued problem''.

%% TODO where is z0? In Omega? Not exactly. In the domain of p(z) and q(z).



%Any point in $\Omega$ can be classified according to the following
%definition.%

%\subsection{Existence  and
%uniqueness theorem of the solution in the neighbourhood of an ordinary
%point\label{subsec:existence}}

%\section{Solution in the neighbourhood of an ordinary point}
%\label{sec:existence}
\begin{theorem}
   Let $z_{0}\in\Omega$ be an ordinary point 
   \graffito{Local existence and uniqueness theorem}
   of Eq.~\eqref{eq:ode}.
   %and let $S\subseteq
   %\Omega$ be the disk centered at $z_{0}$ and whose radius is such that $p(z)$ and
   %$q(z)$ are analytic on $S$.
   Then, for every complex $\alpha$ and $\beta$ there exists one and only one solution of Eq.~\eqref{eq:ode} 
   which satisfies 
   the initial conditions~\eqref{eq:ic} 
   and which is 
   analytic and single-valued  in a suitable neighbourhood of 
   $z_{0}$.
\end{theorem}

\begin{proof}

   Let \graffito{The a-priori knowledge of the disk of convergence of the solution}
   $S$ be the circle with centre at $z_{0}$ and whose radius is  such that every
   point of $S$ is a point of $\Omega$ and is also an ordinary point of the
   Eq.~\eqref{eq:ode}.  There always exists such $S$  with \emph{non-zero} radius
   owing to the fact that $\Omega$ is an open set and all the singularities of
   $p(z)$ and $q(z)$  (if any) are isolated.  We shall prove that a solution of
   Eq.~\eqref{eq:ode} always exists which is \emph{analytic} in $S$ (perhaps also
   in a larger domain) and that such solution is also the unique analytic solution
   of Eq.~\eqref{eq:ode} satisfying the prescribed initial
   conditions~\eqref{eq:ic}.

   The first step is \graffito{First step: equation in ``reduced''
      form} to convert Eq.~\eqref{eq:ode}  into its ``reduced'' form.%
   \footnote{%
      This method of proof is that, \eg, of
      \textcite[\S~10.2]{Whittaker.Watson:1927}.  The method is not suitable
      for generalization to equations of higher orders.  However, the theorem
      holds for linear differential equations of any order.  For a different
      proof which  can be easily adapted to the general case see, \eg,
      \textcite{Smirnov:1964,Wang.Guo:1989}.
   }
   This is done by introducing a new unknown function $v(z)$ related to the
   original one $u(z)$ by 
   \begin{dmath*}
      u(z) = k(z) v(z)
   \end{dmath*},
   where $k(z)$ is an auxiliary function to be determined by requiring
   that the resulting differential equation for $v(z)$ have the coefficient
   function of $v'(z)$
   identically equal to
   zero. We shall see that it always possible to find such $k(z)$ and 
   we shall give an explicit formula for it. 
   In fact, the first- and second-order derivatives of $u(z)$ are
   \begin{dgroup*}
      \begin{dmath*}
	 u'(z) = k'(z) v(z) + k(z) v'(z)
      \end{dmath*},
      \begin{dmath*}
	 u''(z) = k''(z) v(z) + 2k'(z) v'(z) + k(z) v''(z) 
      \end{dmath*}.
   \end{dgroup*}
   Plugging into Eq.~\eqref{eq:ode}, we get the differential equation for $v(z)$, 
   \begin{dmath*}%[label={odev(z)convp(z)}]
      k(z) v''(z) + \big[ 2 k'(z) + p(z) k(z) \big] v'(z) + \big[ k''(z) + p(z)
      k'(z) + q(z) k(z) \big] v(z) = 0
   \end{dmath*},
   and in order the coefficient of $v'(z)$ to be zero, $k(z)$ must satisfy 
   \begin{dmath}[label={odek(z)}]
      2k'(z) + p(z) k(z) = 0
   \end{dmath}.
   Eq.~\eqref{eq:odek(z)} is an homogeneus linear ordinary differential equation of
   the first order in the unknown $k(z)$. One solution of it is the identically
   zero function.  We look for non-identically zero solutions.  One way to get such
   a solution is to rewrite Eq.~\eqref{eq:odek(z)} as
   \begin{dmath*}
      \frac{k'(z) }{k(z)} = - \frac{1}{2} p(z) .
   \end{dmath*}
   The operation of dividing by $k(z)$ is valid because we are looking for non
   identically zero solutions $k(z)$ and if $k(z)$ were zero at some specific point then it
   would be zero everywhere.%
   \footnote{This requires some explanation. It can be seen as  a conseguence of 
      the existence and uniqueness theorem of integrals of first-order linear
      differential equations with given initial conditions. However, although the
      Reader may
      probably be familiar   with this theorem and how to apply it in the context of
      real calculus, here we have never
      stated such theorem in the complex field. To avoid using in the proof some
      fact which we have never dealt with before nor we shall never deal with again in this
      handout, we prefer a more direct approach: We chech a-posteriori that our
      requirement is fulfilled. It is not very elegant but nevertheless it does the
      job!}
   The left-hand side of the previous equation is the logarithmic derivative of
   $k(z)$, therefore by integrating from
   $z_{0}$ to $z\in S$ and exponentiating we get
   \begin{dmath}[label={k(z)}]
      k(z) = \exp \left\{ - \frac{1}{2} \Int{p(\xi)}{\xi, z_{0},z}\right\}
      %\condition*{\forall z \in S}
   \end{dmath},
   up to a constant multiplicative \emph{non-zero} factor which is completely unrelevant for our
   purposes.%
   \footnote{As far as the constant is different from zero, the only effect of such constant
      would be to change the ``initial value''
      $k(z_{0})$, which have no effect on the differential equation since the equation
      is homogeneus. With our choice, $k(z_{0}) = 1$.}
   There is no need to specify in the right-hand side of Eq.~\eqref{eq:k(z)} the
   contour of integration, since the integrand
   $p(\xi)$ is analytic is $S$ (by hypothesis) and $z$ is any point in $S$.
   After dividing by $k(z)$ (now we know from the explicit formula of $k$ that
   $k$ never vanishes) the equation for $v(z)$ becomes  
   \begin{dmath}[label={odev(z)}]
      v''(z) + J(z) v(z) = 0
   \end{dmath},
   where 
   \begin{dmath*}
      J(z) = \frac{k''(z)}{k(z)} + p(z) \frac{k'(z)}{k(z)} + q(z)
   \end{dmath*}.
   We can use Eq.~\eqref{eq:odek(z)} to rewrite $J(z)$ in a more explicit form.
   Notice that
   \begin{dmath*}[compact]
      \left( \frac{k'(z)}{k(z)} \right)^{\prime} = 
      \frac{k''(z) k(z) - \left( k'(z) \right) ^{2} }{k^{2}(z)} = \frac{k''(z)}{k(z)}
      - \left( \frac{k'(z)}{k(z)} \right)^{2} 
   \end{dmath*},
   so
   \begin{dmath*}
      \frac{k''(z)}{k(z)} = \left( \frac{k'(z)}{k(z)} \right)^{\prime} + \left(
	 \frac{k'(z)}{k(z)} \right)^{2} 
   \end{dmath*}.
   Using Eq.~\eqref{eq:odek(z)} we get
   \begin{dmath*}
      \frac{k''(z)}{k(z)} = - \frac{1}{2} p'(z)  + \left( \frac{1}{2} p(z)
      \right)^{2}
   \end{dmath*},
   and then
   \begin{dmath}[label={J(z)}]
      J(z) =  - \frac{1}{2} p'(z) - \frac{1}{4} p^{2} (z) + q(z) 
   \end{dmath}.

   We shall \graffito{From $v(z)$ to $u(z)$} now prove by the method of
   successive approximations that there must exist one and only one solution
   $v(z)$ of Eq.~\eqref{eq:odev(z)} which is analytic and single-valued in $S$
   and which satisfies the initial conditions
   \begin{dmath}[label={icv(z)}]
      %\begin{cases}
      \addtolength{\arraycolsep}{-3pt}
      \left\lbrace
      \begin{array}{rcl}
	 v(z_{0}) &=& c_{0} \\
	 v'(z_{0}) &=& c_{1}
      \end{array}\right. ,
   \end{dmath}
   where $c_{0}$ and $c_{1}$ are arbitrary complex constants.  From this
   result, 
   it will immediately follow that also Eq.~\eqref{eq:ode} must have one and
   only one 
   solution $u(z)$ which is analytic and single-valued in  $S$ and which satisfies
   \begin{dgroup}
      \begin{dmath*}[compact]
	 u(z_{0}) = k(z_{0} ) v(z_{0}) = c_{0} ,
      \end{dmath*}
      \begin{dmath*}[compact]
	 u'(z_{0}) = k'(z_{0}) v(z_{0}) + k(z_{0}) v'(z_{0}) = - \frac{1}{2} p(z_{0})
	 c_{0} + c_{1} .
      \end{dmath*}
   \end{dgroup}
   From the arbitrariness of $c_{0}$ and $c_{1}$,  we will conclude that  there
   must be one and only one solution $u(z)$ of Eq.~\eqref{eq:ode} which is analytic and
   single-valued in $S$ and which satisfies 
   the desired initial
   conditions~\eqref{eq:ic} for any  given  value of $\alpha$ and $\beta$.

   We define 
   \graffito{Second step: construction of the solution}
   a sequence $( v_{n} )_{n\in\N}$ of complex-valued functions on $S$.
   For every $z\in S$, let
   \begin{dmath*}
      v_{0} (z) = a + b \left( z- z_{0} \right) ,
   \end{dmath*}
   where $a$ and $b$ are arbitrary complex numbers,
   and recursively
   \begin{dmath*}
      v_{n+1} (z) = \Int{\left( \xi -z \right) v_{n} (\xi) J(\xi)}{\xi, z_{0}, z}\condition*{n\geq 0 } .
   \end{dmath*}
   At every stage the integrand is an analytic function of $\xi$ and so the
   value of the integral does not depend on the contour of integration.
   Let us prove this fact by induction.
   First of all, $v_{0}(z)$ is easily seen to be analytic for every $z\in S$
   (indeed,
   it is an entire function);
   furthermore, assuming $v_{n}(z)$ to be analytic on $S$ we get 
   that $(\xi -z) v_{n}(\xi) J(\xi) $ as a function of $\xi$ is analytic on $S$  so
   its integral between $z_{0}$ and any $z\in S$ does not depend on the contour of
   integration and defines a new function $v_{n+1}(z)$ which also is analytic on $S$.

   It will be helpful to compute explicitly the first and second derivatives of
   these
   functions.
   For $n=0$ we simply have
   %\begin{dgroup*}
   %\begin{dmath*}
   $v_{0} ' (z)  = b$ and 
   %\end{dmath*}
   %\begin{dmath*}
   $v_{0} '' (z)  = 0$, for every $z\in \Omega$.
   %\end{dmath*}
   %\begin{dsuspend}
   In all other cases we have
   \begin{dmath*}
      v_{n+1}' (z) =  - \Int{ v_{n} (\xi) J(\xi) }{\xi, z_{0},z}
      \condition*{n\geq 0} ,
   \end{dmath*}
   and 
   \begin{dmath*}
      v_{n+1}'' (z) = - v_{n} (z) J(z) 
      \condition*{n\geq 0} ,
   \end{dmath*}
   for every $z\in S$.

   We shall show that
   %\begin{dmath*}
   $\sum_{n=0}^{+\infty} v_{n}(z) $
   %\end{dmath*}
   converges absolutely and uniformly in $S$ to a function $v(z)$ analytic in
   $S$ and which satisfies  the
   Eq.~\eqref{eq:ode}. Further, this function will satisfy the initial
   conditions~\eqref{eq:icv(z)} for any complex $c_{0}$ and $c_{1}$,
   provided that the parameters $a$ and $b$ in the definition of $v_{0}(z)$ have been properly tuned.
   In order to prove these facts, we need a preliminary result, namely, we need to
   show that the following inequality holds:
   %\graffito{A preliminary result to show convergence of the series}
   \begin{dmath}[label={vnbound}]
      \Abs{v_{n}(z)} \leq \mu M^{n} \frac{ \Abs{ z-z_{0}}^{2n}}{n!}
      \condition*{n\geq 0},
   \end{dmath}
   for every $z\in S$, where
   %\begin{dgroup*}
   \begin{dmath*}
      \mu = \sup_{z\in S} \Abs{ v_{0}(z)} ,
   \end{dmath*}
   and
   \begin{dmath*}
      M = \sup_{z\in S} \Abs{J(z)} .
   \end{dmath*}
   (These  definitions make sense because both $v_{0}(z)$ and $J(z)$, being
   analytic functions in $S$, are guaranteed to be 
   bounded in $S$.)
   %\end{dgroup*}
   The rigorous proof of~\eqref{eq:vnbound} is by induction on $n$.
   In fact,
   for $n=0$, we have
   \begin{dmath*}
      \Abs{v_{0}(z)} \leq \mu ,
   \end{dmath*}
   for every $z\in S$, which is obviously verified. 
   Suppose now the validity of Eq.~\eqref{eq:vnbound} for some $n$ and let us
   prove that a similar relation holds for $n+1$.
   By definition,
   \begin{dmath*}
      \Abs{v_{n+1}(z)} = \Abs{ \Int{\left( \xi -z \right) v_{n}(\xi)
	    J(\xi)}{\xi, z_{0},z} }.
   \end{dmath*}
   We have already said that the value of this integral must be the same  no matter
   the contour of integration is chosen. Integrating along the straight line
   joining  $z_{0}$ to $z$, with parametric equation
   \begin{dmath*}
      \xi(t) = z_{0} + (z-z_{0}) t \condition*{t\in[0,1]} ,
   \end{dmath*}
   we have $\xi'(t) = z-z_{0}$ and
   \begin{dmath*}[compact]
      \xi(t)  -z  = z_{0} + \left( z- z_{0} \right) t - z = \left( z-z_{0} \right)
      (t-1) ,
   \end{dmath*}
   thus
   \begin{dmath*}[compact]
      \Int{(\xi-z) v_{n}(\xi) J(\xi)}{\xi, z_{0},z}
      %= \int_{0}^{1} \left[ z_{0} + \left( z-z_{0} \right)t - z \right]  v_{n} (\xi(t)) J(\xi(t))  (z-z_{0}) \udiff{t} 
      = \left( z-z_{0} \right)^{2} 
      \Int{(t-1) v_{n} (\xi(t)) J(\xi(t))}{t,0,1},
   \end{dmath*}
   and its modulus becomes 
   \begin{dmath*}
      \Abs{\Int{(\xi-z) v_{n}(\xi) J(\xi)}{\xi,z_{0},z}} 
      = 
      \Abs{ z-z_{0} }^{2} \Abs{ \Int{(t-1) v_{n} (\xi(t)) J(\xi(t))}{t,0,1}}
      \leq 
      \Abs{ z-z_{0}}^{2} \Int{\Abs{t-1} \Abs{ v_{n}(\xi(t))}
	 \Abs{J(\xi(t))} }{t,0,1} 
      \leq\Abs{ z-z_{0}}^{2n+2} \mu \frac{M^{n+1}}{n!} \Int{(1-t)}{t,0,1}
      \leq\Abs{ z-z_{0}}^{2n+2} \mu \frac{M^{n+1}}{n!} \Int{(1-t) t^{2n}}{t,0,1}
      =\Abs{ z-z_{0}}^{2n+2} \mu \frac{M^{n+1}}{(n+1)!} \Int{(1-t)}{t,0,1}
   \end{dmath*}
   Since $\Abs{z-z_{0}} <\rho$ where $\rho$ is the radius of $S$, we get
   \begin{dmath*}
      \Abs{v_{n}(z)} \leq \frac{1}{n!} \mu M^{n} \rho^{2n}.
   \end{dmath*}
   Now, the numerical series 
   $\sum_{n=0}^{+\infty} \frac{1}{n!} \mu  M^{n} \rho^{2n} $
   is convergent, for its general term is that of the power series expansion of the
   exponential function, so the sum of this series can be written explicitly:
   $\sum_{n=0}^{+\infty} \frac{1}{n!} \mu  M^{n} \rho^{2n} = 
   \mu \E^{M\rho^{2}}$.
   %\footnote{You can use for example the ratio test, but the best way to prove the
   %convergence it of course to notice that its sum is explicitly given by $\mu
   %\uexp^{ MR^{2}}$! It is nothing but the exponential series. }
   It follows that the series of functions $\sum_{n=0}^{+\infty} v_{n}(z)$
   is \emph{absolutely} convergent in  $S$ by Cauchy criterion, furthermore  the M-test of Weierstrass ensures that the
   series is also \emph{uniformly} convergent in $S$.
   Let us denote its sum with
   \begin{dmath*}
      v(z) = \sum_{n=0}^{+\infty} v_{n}(z) .
   \end{dmath*}
   According to the theorem  of Weierstrass on the analiticity of the uniform limit
   of a sequence of analytic functions,
   %\footnote{Let $H(\Omega)$ be the set of all and only those complex-valued
   %functions $f:\Omega \rightarrow \C$ which are analytic on $\Omega$.
   %Then, any sequence $(f_{n})_{n\in\N}$ of analytic functions on $\Omega$ which is
   %uniformly convergent to $f$ on $\Omega$    implies $f\in H(\Omega)$.}
   $v(z)$ is analytic in $S$.

   The last step 
   \graffito{Last step: check that $v(z)$ is a solution}
   it to show that such $v(z)$ actually satisfies
   Eq.~\eqref{eq:odev(z)}.
   Since the series $\sum_{n=0}^{+\infty} v_{n}(z)$ converges uniformly on $S$, 
   we can differentiate it term by term to get
   \begin{dmath*}
      v'(z) = b + \sum_{n=1}^{+\infty} v_{n}'(z) ,
   \end{dmath*}
   and 
   \begin{dmath*}
      v''(z) = \sum_{n=0}^{+\infty} v_{n}''(z) ,
   \end{dmath*}
   for every $z\in S$.
   Using the previously derived formula for $v_{n}''(z)$, we finally get
   \begin{dmath*}[compact]
      v''(z) = - J(z) \sum_{n=0}^{+\infty}  v_{n-1}(z) = -J(z) \sum_{n=0}^{+\infty}
      v_{n}(z) = - J(z) v(z) ,
   \end{dmath*}
   which shows that $v(z)$ satisfies Eq.~\eqref{eq:odev(z)}.

   Obviously, 
   $v(z_{0})=a$ and $v'(z_{0}) = b$. By choosing $a=c_{0}$ and $b=c_{1}$, 
   $v(z)$ satisfies the given initial  
   conditions~\eqref{eq:icv(z)}. The fact that $v(z)$ is also single-valued follows
   from the fact that $S$ is a simple connected domain.

   Let us now address 
   \graffito{Uniqueness}
   the problem of proving  the uniqueness of the solution of
   Eq.~\eqref{eq:odev(z)} satisfying the initial conditions~\eqref{eq:icv(z)}.
   Let $\tilde{v}(z)$ be another such solution and 
   \begin{dmath*}
      w (z) = \tilde{v}(z)  - v(z) ,
   \end{dmath*}
   for every $z\in S$.
   Also $w(z)$ is analytic and single-valued on $S$ and 
   we can easily see by direct computation of its first and second order
   derivatives that it satisfies the same equation of $v(z)$ and
   $\tilde{v}(z)$, \ie,
   \begin{dmath*}
      w''(z) + J(z) w(z) = 0,
   \end{dmath*}
   with the initial conditions $w(z_{0}) = w'(z_{0}) =0$.  By repeated
   differentiation, 
   \begin{dmath*}[compact]
      w'''(z_{0}) = - J'(z_{0}) w(z_{0} ) - J(z_{0}) w'(z_{0} ) = 0,
   \end{dmath*}
   and analogously  we get the general result that $w^{(n)}(z_{0})=0 $ for every
   $n$, \ie,  the derivatives of $w$ of any order vanish at the $z_{0}$.  By
   Taylor's theorem, $w(z)$ is identically zero in $S$, \ie, $v(z) = \tilde{v}(z)$
   for every $z\in S$. 
\end{proof}

Some remarks are in order. \graffito{Are there non-analytic solutions?} First of all,
there still remains the question as to wheater or not there might exist
\emph{non-analytic} solutions of Eq.~\eqref{eq:ode} which satisfy the same
initial conditions~\eqref{eq:ic}, since in the proof of the uniqueness we have
made heavy use of analyticity.  (Of course, if there were such non-analytic
solutions they should be twice differentiable in order to satisfy the
equation.) This question has beeen completely answered in the negative, but we
do not enter the details here.  For further  details, we invite the Reader to
consult the book of \textcite{Ince:1956} and references therein.

The series $\sum_{n=0}^{+\infty} v_{n}(z)$ is not in \graffito{Taylor expansion
   of the solution} general a power series.  However, its sum defines a function
$v(z)$ which, as we have seen, is \emph{analytic} in $S$. This means that we can
Taylor expand $v(z)$ and the Taylor series will be convergent within $S$, and
perhaps also in a larger circle,%
\footnote{As we shall discuss in a deeper way in the \S~\ref{sec:many-valuedness}, the
   singularities of the coefficients of the differential equation need \emph{not}
   in general to
   be singular points also of the integrals of the differential equation.}
and
the same is true also for $u(z)$.
In principle, one
can determine $u(z)$ by constructing explicitly the sequence $v_{n}(z)$,
then evaluating in closed form the sum of the series and using $u(z) =k(z) v(z)$ where $k(z)$ is
given by Eq.~\eqref{eq:k(z)}. In practise, it would
become hard or impossible to perform in closed form all steps of such computation.  But now that we know
that one and only one solution $u(z)$ of Eq.~\eqref{eq:ode} exists and is
\emph{analytic}  in $S$, we can find it by directly 
looking for its  series representation with center  at $z_{0}$:
\begin{dmath*}
   u(z) =  \sum_{n=0}^{+\infty} A_{n} (z-z_{0})^{n} .
\end{dmath*}
A priori, we know that  the series will be convergent \emph{at least} within
$S$.  The parameters $A_{0}$ and $A_{1}$ will be  fixed by the initial
conditions~\eqref{eq:ic}, in fact $u(z_{0}) = A_{0}$ and $u'(z_{0}) = A_{1}$;
the other coefficients are found by plugging the above series representation
into Eq.~\eqref{eq:ode} and equating the coefficients of successive powers of
$(z-z_{0})$ to zero.  The power series represenations of the first  and second
order derivatives of $u(z)$ are simply obtained by differentiating the above sum
term by term, since the series is a priori guaranteed to converge uniformly
within $S$.  Notice that in pluggin the series expansions  in the differential equation, it might be easier to work
directly with the differential equation cleared of fractions, without having to
reduce it to the standard form Eq.~\eqref{eq:ode} in which the coefficient of
the second-order derivative has been put equal to one. The standard form
Eq.~\eqref{eq:ode} is not always the best one to work with at a practical level,
but it is quite adeguate for theoretical manipulations.

Finally, \graffito{The a priori knowledge of the disk $S$} notice that the results of the
theorem of this section hold only locally, \ie, in the  neighbourhood $S$ of
$z_{0}$.  But $S$ is known \emph{a priori}, \ie, it does not depend on the
initial conditions and therefore it is the same for \emph{all} the integrals of
Eq.~\eqref{eq:ode} about the ordinary point $z_{0}$.  In this case we speak of
\emph{fixed} singularities.




For 
\graffito{Movable singularities may occur
   in the case of non-linear equations}
\emph{non-linear} equations, it may happen that the disk of convergence depends on
the initial conditions. Consider for instance the first-order
homogeneus non-linear differential equation
\begin{dmath*}
   u'(z) + u^{2}(z) = 0,
\end{dmath*}
whose non identically-zero general integral is given by
\begin{dmath*}
   u(z) = \frac{1}{z - c} ,
\end{dmath*}
where the arbitrary constant of integration $c$  has to be fixed by imposing
initial conditions at some point $z_{0}$:
\begin{dmath*}
   u(z_{0}) = \alpha ,
\end{dmath*}
with $\alpha \neq 0$.
(If $\alpha =0$, the solution would be $u(z) =0$ everywhere.)
By direct substitution we get
$c = z_{0} - \frac{1}{\alpha} $,  so
\begin{dmath*}
   u(z) = \frac{1}{(z-z_{0}) + \frac{1}{\alpha}} .
\end{dmath*}
There is always a disk of non-zero radius with center at $z_{0}$ in which this
function is analytic and single-valued and has no singular points.  However, the
position of its singularity depends on the initial condition, for in fact $u(z)$
has an isolated singular point when $z=z_{0} - \frac{1}{\alpha}$, which can be
made closer and closer to $z_{0}$ and which cannot be located a priori.  In this
case, one speaks of  ``movable'' singularities.


%The example of 

\section{The Wronskian}

In order to make the presentation as self contained as possible, it seeems
useful to briefly  review 
some material concerning the ``Wronskian'' of two integrals of the differential
equation. 
%This section is mostly taken from \textcite{Sansone.Gerretsen:1969}.


%Recall \graffito{Linear independence of vectors} from linear algebra that if $V$ denotes some vector space over the field
%of complex numbers $\C$, then two elements $\vec{u}$ and $\vec{v}$ are defined
%to be ``linearly \emph{in}dependent'' if  the  linear combination
%\begin{dmath*}
%\vec{w} = 
%\alpha \vec{u} + \beta\vec{v} 
%\end{dmath*} with complex coefficients 
%$\alpha$ and $\beta$, is zero \emph{if and only if} 
%$\alpha = \beta =0$. 
%Otherwise, the two vectors are said to be ``linearly dependent'', which means
%that  there exist $\alpha$ and $\beta$, with $\alpha $ and $\beta$ \emph{not}
%both zero,  such that  
%$\vec{w}=\vec{0}$. (Here, $\vec{0}$ denotes the null vector, while $0$ denotes
%the number zero in $\C$.)

As usual, \graffito{Linear independence of two analytic functions} we define two
functions $w_{1}$ and $w_{2}$, defined and
analytic in the same domain $\Omega$, to be  ``linearly \emph{in}dependent'' if the linear
combination
\begin{dmath}[label={li}]
   w(z) = \alpha w_{1}(z) + \beta w_{2}(z) 
   \condition*{\forall z\in \Omega}
\end{dmath},
where $\alpha$ and $\beta$ are complex numbers, vanishes identically  in
$\Omega$ \emph{if and only if} $\alpha = \beta =0 $. 
Otherwise, $w_{1}$ and $w_{2}$ are said to be ``linearly dependent'', which means
that it is possible to find $\alpha$ and $\beta$ such that 
\begin{dmath*}
   \alpha w_{1}(z) + \beta w_{2}(z)  =0
\end{dmath*}
identically and at least one of the two complex numbers $\alpha$ and $\beta$ is
different from zero.
(It is nothing more than
the usual notion of linear independence in linear algebra.)


Let \graffito{Definition of the Wronskian} us introduce the ``Wronski
determinant'' or ``Wronskian'' of $w_{1}$ and
$w_{2}$; it is defined by
\begin{dmath*}[compact]
   W_{w_{1}, w_{2}} (z) = 
   \det \begin{pmatrix}
      w_{1} (z) & w_{2} (z) \\
      w_{1}'(z) & w_{2}'(z) 
   \end{pmatrix} 
   = w_{1}(z) w_{2}'(z) -w_{1}'(z) w_{2}(z) 
\end{dmath*}.
Notice that it is  is a function of $z\in\Omega$.

The Wronskian is useful  to test linear independence.
In fact, we have the following theorems.

\begin{theorem}
   If $w_{1}$ and $w_{2}$ are linearly dependent, then $W_{w_{1}, w_{2}}$ vanishes
   identically.
\end{theorem}

\begin{proof}
   Differentiation of Eq.~\eqref{eq:li} yields
   \begin{dmath*}
      w'(z) = \alpha w_{1}'(z) + \beta w_{2}'(z) 
   \end{dmath*}.
   By hypothesis, 
   $w_{1}$ and $w_{2}$ are linearly dependent, thus
   $w(z)$ is identically zero.
   Since $w(z)$ is constant in $\Omega$, $w'(z) = 0$ for all $z\in\Omega$.
   In matrix form, 
   \begin{dmath*}
      \begin{pmatrix} 
	 w_{1}(z) & w_{2} (z) \\
	 w_{1}'(z) & w_{2}'(z)
      \end{pmatrix}
      \begin{pmatrix}
	 \alpha \\ \beta 
      \end{pmatrix} = \begin{pmatrix} 0 \\ 0 \end{pmatrix}
   \end{dmath*}.
   This system of equations is homogeneus, thus $\alpha = \beta =0$ is a solution.
   Since by hypothesis, 
\end{proof}

The converse of the previous theorem is in general \emph{not} true, \ie, it
might happen that the Wronskian of $w_{1}$ and $w_{2}$ is zero \emph{and}
$w_{1}$ and $w_{2}$ are linearly \emph{in}dependent  in their domain of
definition.

\begin{approfondimento}
   Let us consider an example with differentiable functions of a real variable.
   Let $f_{1}, f_{2} : \R\rightarrow \R
   $ be defined by 
   \begin{dgroup*}
      \begin{dmath*}
	 f_{1}(t) = t^{2}
      \end{dmath*}
      \begin{dsuspend}
	 and
      \end{dsuspend}
      \begin{dmath*}
	 f_{2}(t) = t\Abs{t}
      \end{dmath*},
   \end{dgroup*}
   for every $t\in\R$. The two functions are differentiable and linearly
   \emph{in}dependent on $\R$, for  
   there exists no $\lambda \in\R$ such that $f_{1}(t) =
   \lambda f_{2}(t)$ for every $t\in\R$.
   But their  Wronskian is identically zero for all $t\in\R$. 
\end{approfondimento}

\section{Many-valuedness and singularities of the solutions}
%{Investigation of
%the many-valuedness and singularities of the solutions}
\label{sec:many-valuedness}

\subsection{Analytic continuation of the solution}

We stress once again that the theorem of \S~\ref{sec:existence} is a
\emph{local} result, in the sense that corresponding to every ordinary point
$z_{0}$ of the Eq.~\eqref{eq:ode} a solution  of Eq.~\eqref{eq:ode} satisfying
the initial conditions~\eqref{eq:ic} for arbitrarily-assigned complex values of
$\alpha$ and $\beta$ must exist in $S$ and therein is guaranteed to be  uniquely
determined, analytic and single-valued.  However, the theorem of
\S~\ref{sec:existence} gives no information regarding the behavior of the
solution at points of $\Omega$ away from $S$.  In other words, the result only
applies to a neighbourhood of the ordinary point $z_{0}$.

By the \graffito{Analytic continuation of the solution} general process of
analytic continuation, the series representing the solution in $S$ can be
analytically-continued  at all points of $\Omega$ with the exception of those
which are singular points of the coefficients of the equation.  The actual
implementation of this procedure is explain in some detail in figure.

\begin{theorem}
   The analytic continuation of a solution of Eq.~\eqref{eq:ode} is also a solution
   of Eq.~\eqref{eq:ode}.
\end{theorem}

\begin{proof}

   Suppose $w(z)$ is a solution of Eq.~\eqref{eq:ode} in the circle $\Abs{ z
      -z_{0}} \leq \rho$ and let $w^{*} (z)$ be the analytic continuation of $w(z)$ in
   some other circle $\Abs{ z-z_{0}}\leq \tilde{\rho}$ sharing a region $G$ with
   the previous. Consider function 
   \begin{dmath*}
      F(z) = w''(z) + p(z) w'(z) + q(z) w(z) ,
   \end{dmath*}

\end{proof}

Therefore, \graffito{The a priori knowledge of the possible singular points of
   the integrals of a linear differential equation} we have obtained a solution of
Eq.~\eqref{eq:ode} satisfying the initial conditions~\eqref{eq:ic} for every
complex $\alpha$ and $\beta$, which is \emph{analytic} at \emph{all} points of
$\Omega$ except, eventually, those points of $\Omega$ which are singular points
of the coefficients of the equation. The construction does not work for those
points, but this does not mean that the series representing the solution cannot
converge also at those points. In general, the singular points of the
coefficients \emph{may} be singular points of \emph{some} integrals of the
equation.  But any integral of Eq.~\eqref{eq:ode} cannot have singularieties at
points of $\Omega$ other than those which are singular points of the
coefficients.

Although any integral of Eq.~\eqref{eq:ode} is guaranteed to be \emph{analytic}
in $\Omega$ except at (perhaps) the singularities (if any) of the coefficients
of the equation, the key point is that in general these integrals need not to be
\emph{single-valued} throughout $\Omega$ even though the coefficients of the
equation are supposed to be single-valued.  We shall now turn to investigate
this problem.


\subsection{Local solutions near a singular point}

The main result of this section can be summarized  in the following theorem.

\begin{theorem}
   Let $z_{0} \in \Omega$ be a \emph{singular} point of eq.~\eqref{eq:ode}.
   Let $S$ be the annulus with center at $z_{0}$ and non-zero radius in which
   the Laurent series of 
   $p(z)$ and $q(z)$ about $z_{0}$ are convergent.
   Two cases are possible:
   \begin{aenumerate}
   \item There exist two 
   linearly \emph{in}dependent integrals of eq.~\eqref{eq:ode}  which are given by
   \begin{dmath*}
      w_{1} (z) = (z-z_{0})^{\rho_{1}} \phi_{1}(z)
   \end{dmath*},
   and 
   \begin{dmath*}
      w_{2} (z) = (z-z_{0})^{\rho_{2}} \phi_{2}(z)
   \end{dmath*},
   where $\rho_{1}$ and $\rho_{2}$ are complex numbers such that
   $\rho_{1}-\rho_{2}\notin\Z$ and $\phi_{1}$ and $\phi_{2}$ are analytic in
   the annulus $S$.
   \item
   There exist two linearly \emph{in}dependent  
   integrals of eq.~\eqref{eq:ode}, one is again given by 
   \begin{dmath*}
      w_{1} (z) = (z-z_{0})^{\rho_{1}} \phi_{1}(z)
   \end{dmath*},
   while the other is given by
      \begin{dmath*}
	 w_{2} (z) = w_{1}(z) \left[ A \log (z-z_{0}) + \psi(z) \right]
      \end{dmath*},
      where $A\in \C$ and $\psi(z)$ is  analytic in the annulus $S$. 
   \end{aenumerate}
   $\rho_{1}$, $\rho_{2}$ are the solutions of the so-called ``foundamental
   equation'' associated with eq.~\eqref{eq:ode} and the singular point $z_{0}$.
\end{theorem}


Let $z_{0}\in\Omega$ be a singular point of the coefficients of the
eq.~\eqref{eq:ode} and $\sigma$ be any neighbourhood of $z_{0}$ all of whose
points are ordinary points of Eq.~\eqref{eq:ode}, \eg, $\sigma$ can be the disk
with centre at $z_{0}$ and having non-zero radius.  The fact that such
neighbourhood always exists follows from  the previous hypothesis that $\Omega$
is an open set and the singularities of the coefficients are isolated.  Further,
let $b$ any ordinary point of $\sigma$  and $\Gamma$ any simple closed contour,
beginning and ending at $b$, which lies completely within $\sigma$ (\ie, not
passing through any singularity of the equation and such that there is no
singularity of the equation other than $z_{0}$ in its interior) and which
encircles exactly once in the positive direction (with respect to the area
enclosed by it) the singular point $z_{0}$.

Consider two arbitrary linearly independent solutions $u_{1}(z)$ and $u_{2}(z)$
in the neighbourhood of $b$.  Their corresponding analytic continuations along
$\Gamma$ will be denoted by $\tilde{u}_{1}$ and $\tilde{u}_{2}$.

Since the coefficients $p(z)$ and $q(z)$ are left unaltered by the description
of $\Gamma$ (because they are single-valued), by the previous theorem of this
section we get that at all stages of the analytic continuation along $\Gamma$
the Eq.~\eqref{eq:ode} remains identically fulfilled, in particular, also
$\tilde{u}_{1} $ and $\tilde{u}_{2}$ are solutions of Eq.~\eqref{eq:ode}.

Further, the analytic continuations of two linearly independent solutions are
linearly independent solutions.
This can be easily seen by considering the complex analogous of the so-called
Abel's formula (or also Liouville's formula) for the Wronski determinant.


The Wronski determinant
\graffito{Wronski determinant} (the so-called Wronskian) of any two
differentiable functions
$u_{1}(z)$ and $u_{2}(z)$ is defined by 
\begin{dmath}[label={wronskian},compact]
   W[u_{1}, u_{2}](z) = \det \begin{pmatrix} 
      u_{1} (z) & u_{2}(z) \\
      u_{1}'(z) & u_{2}'(z) \end{pmatrix}  .
   %= u_{1}(z) u_{2}'(z) - u_{1}'(z) u_{2}(z) .
\end{dmath}
Notice that it is a function of $z$. The Wronski determinant is useful
to test linear independence. In fact,
if $u_{1}$ and $u_{2}$ are 
\graffito{Sufficient condition for linear dependence}
linearly dependent, this means that there must exist some complex constant $\lambda$
such that $u_{1}(z) = \lambda
u_{2}(z)$, then $u'_{1} (z) = \lambda u'_{2} (z)$ and
the Wronskian  is
identically zero in its domain of definition.
The
converse in general is not true,
%\footcite[See, \eg,][]{Bocher:1900}
but for analytic functions identical vanishing
of the Wronskian implies the linear dependence of the functions.


%\begin{dmath*}
%f_{1}'(t) = 2t 
%\end{dmath*}
%and
%
For the Wronski determinant of integrals of a linear differential equation, it
is easy to prove a formula known as Abel or Liouville formula which gives (up to
a constant) the value of the Wronskian for any two integrals only from the
knowledge of the differential equation!
Let us now derive this formula.
Suppose that $u_{1}(z)$ and $u_{2}(z)$ are solutions of
Eq.~\eqref{eq:ode} in some domain. 
Then
\begin{dmath*}
   u_{1} ''(z) + p(z) u_{1} '(z) + q(z) u_{1}(z) =0 ,
\end{dmath*}
and 
\begin{dmath*}
   u_{2} ''(z) + p(z) u_{2} '(z) + q(z) u_{2}(z) =0 .
\end{dmath*}
Multiplying the first equation by $u_{2}(z)$ and  the second one by $u_{1}$ and
subtractin one from the other we get 
\begin{dmath}
   W'[u_{1}, u_{2}](z) -p(z) W[u_{1},u_{2}](z) =0 ,
\end{dmath}
which is a linear homogeneus ordinary differential  equation of the first order
for the Wronskian. (The first term on the left-hand side is the derivative in
the complex sense of the Wronskian with respect to $z$, remember that
$W[u_{1},u_{2}]$ is a function of $z$.)
By integrating from $z_{0}$ to $z$ (assuming that the coefficient $p(z)$ is
analytic) we get
\graffito{Abel formula}
\begin{dmath}
   W[u_{1}, u_{2}](z) = 
   W[u_{1}, u_{2}](z_{0}) \exp \left\{ - %\int_{z_{0}}^{z} p(\xi) \udiff{\xi}
      \Int{p(\xi)}{\xi,z_{0},z}
   \right\}.
\end{dmath}
This is the well-known Abel, or Lioville, formula for the Wronski determinant of
integrals of Eq.~\eqref{eq:ode}.
An analogous formula holds for linear equations of arbitrary 
order, in that case the integral of $p(\xi)$ has to be replaced by the
coefficient of the derivative of the $(n-1)$\ordth. This formula also holds in
the real case and  indeed it is easier  to derive it  in the real case
(since you have not to worry as weather or not it is necessary to prescribe some
contour of
integration, etc, as it happens in the complex case).

Therefore, there must exist four complex constants $\alpha$, $\beta$, $\gamma$,
$\delta$ such that 
\begin{dmath*}
   \tilde{u}_{1}(z) =  \alpha u_{1}(z) + \beta u_{2}(z), 
\end{dmath*}
and 
\begin{dmath*}
   \tilde{u}_{2}(z) =  \gamma u_{1}(z) + \delta  u_{2}(z), 
\end{dmath*}
with
\begin{dmath*}
   \det \begin{pmatrix} \alpha & \beta \\ \gamma & \delta \end{pmatrix} \neq 0 ,
\end{dmath*}
otherwise $\tilde{u}_{1}$ and $\tilde{u}_{2}$ would be linearly dependent, in
contrast with the earlier hypothesis that $u_{1}$ and $u_{2}$ are chosen
linearly \emph{in}dependent.

It can be shown, see \textcite{Tricomi:1961}, that the constants $\alpha$,
$\beta$, $\gamma$, $\delta$ depend in general on the choice of the linearly
independent integrals $u_{1}$ and  $u_{2}$ but not on the particular contour
$\Gamma$ considered.


Among all integrals of Eq.~\eqref{eq:ode}, we look for integrals $w(z)$ which
undergo the simplest of all possible linear transformation on  describing the contour $\Gamma$,
namely,
\begin{dmath}[label={lambda}]
   \tilde{w}(z) = \lambda w(z) ,
\end{dmath}
for some complex $\lambda$, where $\tilde{w}$ denotes the integral obtained by
analytically continuing $w$ along $\Gamma$.
Clearly, not all integrals of Eq.~\eqref{eq:ode} are of this type. 
If there were solutions of Eq.~\eqref{eq:ode} of this form, then it would be
possible to express them as a linear combination of $u_{1}$ and $u_{2}$, at
least locally, \ie, there must exist complex coefficients
$k_{1}$ and $k_{2}$ such that 
\begin{dmath*}
   w(z) = k_{1} u_{1}(z) + k_{2} u_{2}(z) .
\end{dmath*}
Since the analytic continuation of a linear combination of functions is the
linear combinations of the analytic continued functions, we get 
\begin{dmath*}
   \tilde{w}(z) = k_{1} ( \alpha u_{1} (z) + \beta u_{2} (z) ) + k_{2} ( \gamma
   u_{1}(z) + \delta u_{2}(z) ) 
   = \left( \alpha k_{1} + \gamma  k_{2} \right) u_{1} (z) + 
   \left( \beta k_{1} + \delta k_{2} \right) u_{2}(z) .
\end{dmath*}
It follows that Eq.~\eqref{eq:lambda} is satisfied if and only if 
\begin{dgroup}
   \begin{dmath*}
      k_{1} \alpha + k_{2} \gamma = \lambda k_{1} ,
   \end{dmath*}
   \begin{dmath*}
      k_{1} \beta + k_{2} \delta = \lambda k_{2} ,
   \end{dmath*}
   \begin{dsuspend}
      that is
   \end{dsuspend}
   \begin{dmath*}
      k_{1} \left( \alpha -\lambda \right) + k_{2} \gamma = 0
   \end{dmath*}
   \begin{dmath*}
      k_{1} \beta + k_{2} \left( \delta - \lambda \right) = 0 .
   \end{dmath*}
\end{dgroup}
This system of two simulataneus homogeneus equation has a trivial solution
$k_{1}= k_{2} = 0$ yielding a identically-zero integral $w$.
We look for non trivial solutions.
The system has non trivial solutions if 
\begin{dmath}[label={ce}]
   \det \begin{pmatrix} \alpha - \lambda & \gamma \\ \beta & \delta - \lambda
   \end{pmatrix} = 0 .
\end{dmath}
Eq.~\eqref{eq:ce} is called ``characteristic'' or the ``foundamental''
equation associated with the singular point $z_{0}$.
It is a quadratic equation for $\lambda$ which has two \emph{non-zero}%
\footnote{Their product is $\lambda_{1} \lambda_{2} = \alpha \delta - \beta
   \gamma$, which must be non-zero by earlier considerations.}
complex roots
$\lambda_{1}$ and $\lambda_{2}$, which eventually may coincide.
This equation restricts the possible values that $\lambda$ can have in
Eq.~\eqref{eq:lambda}.

It can be shown that while $\alpha$, $\beta$, $\gamma$ and $\delta$ depend on
the choice of $u_{1}$ and $u_{2}$, the roots of the characteristic equation are
independent of this choice and depend only on the coefficients of
Eq.~\eqref{eq:ode} and on the point $z_{0}$.  The proof is simple but rather
tedius, see, \eg, \textcite[\S~15.20]{Ince:1956}.  

In the first place, 
\graffito{Non-repeated roots} let us consider the case when the characteristic
equation has two distinct roots, \ie, $\lambda_{1} \neq
\lambda_{2}$.
The system now reduce to only one equation which determines 
the values of $k_{1}$ and $k_{2}$ and thus (apart from a
factor of proportionaly) two integrals $w_{1}(z)$ and $w_{2}$ which
satisfy Eq.~\eqref{eq:lambda} for $\lambda=\lambda_{1}$ and
$\lambda=\lambda_{2}$ respectively.

We shall prove now that $w_{1}$ and $w_{2}$ must be linearly \emph{in}dependent.
Otherwise, the function $w_{1} / w_{2}$ would be a constant, but this is
impossible since a constant would  not change on describing $\Gamma$ while
$w_{1}/w_{2}$ acquires a factor $\lambda_{1} / \lambda_{2}$ in doing so.
Therefore, we have found two linearly \emph{in}dependent  solutions $w_{1}$ and
$w_{2}$ and we may work with them instead of using the original $u_{1}$ and
$u_{2}$. The reason to prefer $w_{1}$ and $w_{2}$ is that, as we shall see in a
while, they have a particular simple form.

Consider the function $\left( z - z_{0} \right)^{\rho}$ which, on describing the
contour $\Gamma$ becomes $\E^{2\pi i \rho } \left( z - z_{0} \right)^{\rho}$,
\ie, acquires a multiplicative factor $\E^{ 2\pi i \rho}$ which in equal to
one if and only if $\rho $ is an integer number.
Suppose to select a $\rho_{1}$ so that 
\begin{dmath*}
   \E^{2\pi i \rho_{1} } = \lambda_{1} ,
\end{dmath*}
and analogously 
\begin{dmath*}
   \E^{2\pi i \rho_{2} } = \lambda_{2} ,
\end{dmath*}
which means that 
\begin{dgroup*}
   \begin{dmath*}
      \rho_{1} =  \frac{1}{2\pi i } \Log \lambda_{1} 
   \end{dmath*},
   \begin{dmath*}
      \rho_{2} =  \frac{1}{2\pi i } \Log \lambda_{2} 
   \end{dmath*},
\end{dgroup*}
then the two functions
\begin{dgroup}[label={phi}]
   \begin{dmath}
      \phi_{1}(z) = \frac{w_{1}(z)}{\left( z-z_{0} \right)^{\rho_{1}}} 
   \end{dmath},
   \begin{dmath}
      \phi_{2}(z) = \frac{w_{2}(z)}{\left( z-z_{0} \right)^{\rho_{2}}} 
   \end{dmath},
\end{dgroup}
remains single-valued on describing the singular point $z_{0}$.
Thus, the two functions $\phi_{1,2}(z)$ are analytic and single-valued in a disk
centered at $z_{0}$ (except the point $z_{0}$) and can thus be expanded in
Lauren series.
Thus, 
\begin{dgroup}[label={phiseries}]
   \begin{dmath}
      w_{1}(z) = \left( z - z_{0} \right)^{\rho_{1}} \sum_{n \in \Z} a_{n} \left( z -
	 z_{0} \right)^{n} 
   \end{dmath},
   \begin{dmath}
      w_{2}(z) = \left( z - z_{0} \right)^{\rho_{2}} \sum_{n \in \Z} b_{n} \left( z -
	 z_{0} \right)^{n} 
   \end{dmath}.
\end{dgroup}


%\subsection{The a priori  knowledge of the singular points of the integrals of a
%linear differential equation}



%\section[Regular singular points]{Solution in the neighbourhood of a singular point. Regular singular
%points}

\section{Fuchs theorem}

\begin{sidefigure}
   \includegraphics[width=\marginparwidth]{Lazarus_Immanuel_Fuchs}
   \caption{Lazarus Fuchs}
\end{sidefigure}

There is a necessary and sufficient criterion, due to L.~Fuchs, which enables us to establish  \emph{a priori} whether or not a singular point of
Eq.~\eqref{eq:ode} is a Fuchsian singular point directly from the inspection  of
the differential equation. This is the content of Fuchs' theorem. The proof of
the theorem will lead us to a practical way to determine a foundamental system
of solutions in a neighbourhood of a Fuchsian singular point.

\begin{theorem}
   A %\graffito{Fuchs' necessary and sufficient characterization of regular singular
   %points} 
   singular point 
   $z_{0}\in \Omega$ is a Fuchsian singular point of Eq.~\eqref{eq:ode} \emph{if
      and only if} 
   \begin{itemize}
      \item $p(z)$ has \emph{at most} a pole of the \emph{first} order at the point
	 $z_{0}$ and
      \item $q(z)$ has \emph{at most} a pole of the \emph{second} order at the point
	 $z_{0}$;
   \end{itemize}
   \ie, \emph{if and only if}
   $(z-z_{0}) p(z)$ and $(z-z_{0})^{2} q(z)$ are \emph{analytic} at the point
   $z_{0}$.
\end{theorem}

%\begin{proof}

\subsection{Proof of the necessity of Fuchs' conditions}


Let us first establish the necessity of Fuchs' conditions.
We shall need the following lemma.
\begin{lemma}\label{lemma:classA}
   Let 
   \begin{dmath}[label={classA}]
      f(z) = \left( z -z_{0} \right)^{\alpha} g (z) 
   \end{dmath},
   where $\alpha$ is \emph{any} complex number and $g(z)$ is analytic at $z_{0}$.
   Then 
   \begin{itemize}
      \item 
	 $\frac{f'(z)}{f(z)}$ has \emph{at most} a pole of the \emph{first} order at
	 $z_{0}$;
      \item $\frac{f''(z)}{f(z)}$ has \emph{at most} a pole of the
	 \emph{second}
	 order at $z_{0}$.
   \end{itemize}
\end{lemma}

\begin{proof}

   The case in which $g$ vanishes identically (and thus $f$) is trivial, so
   let us assume that $g$ does not vanish identically.

   \begin{approfondimento}
      If $g(z_{0}) = 0$ and since $g$ is assumed to be not identically zero,
      then there must exist $n\in\N$ such that
      \begin{dmath*}
	 g(z) = \left( z-z_{0} \right)^{n} \tilde{g} (z)  
      \end{dmath*},
      for every $z$ in a neighbourhood of $z_{0}$, 
      $\tilde{g}$ being  
      \emph{analytic} and different from zero in such
      neighbourhood (including the point $z_{0}$), \ie, 
      $g$ has a zero of ``order'' $n$ at $z_{0}$.
   \end{approfondimento}


   Let us suppose with great generality that $g$ has a zero of order $n\in\N$ at $z_{0}$.
   (Eventually, $n=0$.)
   Then, we may write
   \begin{dmath*}
      f(z) = \left( z- z_{0} \right)^{\alpha+n} \tilde{g} (z) 
   \end{dmath*},
   where $\tilde{g}(z) \neq 0$ in the neighbourhood of $z_{0}$.
   By differentiating, we get
   \begin{dgroup*}
      \begin{dmath*}
	 \frac{f'(z) }{f(z)} = \frac{\alpha + n}{z - z_{0} } + \frac{\tilde{g}'(z)}{g(z)} 
      \end{dmath*}
      \begin{dmath*}
	 \frac{f''(z)}{f(z)} = \frac{(\alpha + n ) (\alpha + n - 1)}{\left( z-z_{0}
	    \right)^{2} } + \frac{2 ( \alpha + n )}{z -z_{0} }
	 \frac{\tilde{g}'(z)}{\tilde{g}(z)} +
	 \frac{\tilde{g}''(z)}{\tilde{g}(z)}
      \end{dmath*}.
   \end{dgroup*}
   Since $\tilde{g}$ never vanishes and $\tilde{g}'$ and $\tilde{g}''$
   are analytic at $z_{0}$ (for
   $\tilde{g}$ is analytic at $z_{0}$), we conclude that 
   $\frac{f'(z)}{f(z)}$ has \emph{at most} a pole of the first order at
   $z_{0}$,%
   \footnote{%
      We say at most
      because it may happen that $\alpha + n =0 $.
   } and
   $\frac{f''(z)}{f(z)}$ has \emph{at most} a pole of the second order at
   $z_{0}$.%
   \footnote{%
      We say at most because it may happen that either $\alpha + n =0 $ or $\alpha + n
      -1=0$.
   }
\end{proof}

\begin{remark}
   Products, quotients and derivatives of functions of the
   form~\eqref{eq:classA} are also  of the form~\eqref{eq:classA}.  Sums of
   functions of the form~\eqref{eq:classA} in general are not of the
   form~\eqref{eq:classA}, except in the special case in which the corresponding
   indices $\alpha$ differ by an integer number.
\end{remark}


Our goal is to prove that if $z_{0}$ is a Fuchsian singular point of
eq.~\eqref{eq:ode}, then $p(z)$ and $q(z)$ have,  at most, a pole of the first
and second order at $z_{0}$, respectively.  In order to prove that, we write
$p(z)$ and $q(z)$ in terms of $w_{1}(z)$ and $w_{2}(z)$.  We have
\begin{dgroup*}
   \begin{dmath}[label={nfuchs:p(z)}]
      \Style{DDisplayFunc=outset}
      p(z) = - \D{\log W_{w_{1}, w_{2}}(z)  }{z}
      = -\D{\log  \left( w_{1}(z) w_{2}'(z) - w_{1}'(z) w_{2}(z) \right)}{z} 
      = - \D{ \log w_{1}^{2} (z)  \left( \frac{ w_{1}(z) w_{2}'(z) - w_{1}'(z)
	       w_{2}(z)}{w_{1}^{2}(z)} \right)}{z} 
      = - \D{ \log \left\lbrace w_{1}^{2}(z) \D{ \left( \frac{w_{2}(z)}{w_{1}(z)}
	    \right)}{z} \right\rbrace}{z} 
      \Style{DDisplayFunc=inset}
   \end{dmath},
   \begin{dsuspend}
      and 
   \end{dsuspend}
   \begin{dmath}[label={nfuchs:q(z)}]
      q(z) = - \frac{w_{1}''(z)}{w_{1}(z)} - p(z) \frac{w'_{1}(z)}{w_{1}(z)} 
   \end{dmath}.
\end{dgroup*}



We have to consider two cases. Let us first consider the case in which  
$w_{1}(z)$ and $w_{2}(z)$ are of the
form~\eqref{eq:classA} [see Eqs.~\eqref{eq:nfuchs:w(z)}] and so are
\begin{dseries*}
   \begin{math}
      \frac{w_{2}(z)}{w_{1}(z)}
   \end{math}, 
   \begin{math}
      \Style{DDisplayFunc=outset}
      \D{\frac{w_{2}(z)}{w_{1}(z)}}{z}
      \Style{DDisplayFunc=inset}
   \end{math},
   \begin{math}
      \Style{DDisplayFunc=outset}
      w_{1}^{2}(z) \D{\frac{w_{2}(z)}{w_{1}(z)}}{z}
      \Style{DDisplayFunc=inset}
   \end{math}
\end{dseries*}
and the logarithmic derivative of the last function is equal to $-p(z)$ and has
at most a pole of the first order, which implies that $p(z)$ has at most a pole
of the first order at $z_{0}$.

What is about the logarithmic case, \ie, when $w_{2}$ is given by

\subsection{Proof of the sufficiency  of the Fuchs' conditions}

We now turn to prove that Fuchs' conditions are also sufficient.
By hypothesis, the functions $P(z) = \left( z- z_{0} \right)p(z)  $ and
$Q(z) = \left( z- z_{0} \right)^{2} q(z)$ are analytic at $z_{0}$ and therefore
\begin{dgroup}
   \begin{dmath}[label={P(z)}, compact]
      P(z) =  \left( z- z_{0} \right) p(z) = \sum_{n=0}^{+\infty} p_{n} \left( z -
	 z_{0} \right)^{n} 
   \end{dmath},
   \begin{dmath}[label={A(z)}, compact]
      Q(z) =  \left( z- z_{0} \right)^{2} q(z) = \sum_{n=0}^{+\infty} q_{n} \left( z -
	 z_{0} \right)^{n} 
   \end{dmath}.
\end{dgroup}
Notice that we do not exclude the possibility that some of the leading
coefficients in these two Taylor series expansions may be zero.

We look for solutions of the form
\begin{dmath}
   w(z) = \left( z -z_{0} \right)^{\rho} \sum_{n=0}^{+\infty} w_{n} \left( z-z_{0}
   \right)^{n} 
\end{dmath},
where $\rho$ and the coefficients $w_{n}$ (with $n\in\N$) are to be determined.
We can always suppose that $w_{0} \neq 0$, for if $w_{0}$ were zero, we could in
any case shift the series since $\rho$ is unknown.

Let us write Eq.~\eqref{eq:ode} in the form
\begin{dmath}[label={odeP}]
   \left( z - z_{0} \right)^{2} w''(z) + \left( z - z_{0} \right) P(z) w'(z) + Q(z)
   w(z) = 0
\end{dmath}.
The derivatives of $w(z)$ are 
\begin{dgroup*}
   \begin{dmath*}
      w'(z) = \rho \left( z-z_{0} \right)^{\rho -1} \sum_{n=0}^{+\infty} w_{n} \left(
	 z -z_{0} \right)^{n} + \left( z- z_{0} \right)^{\rho} \sum_{n=0}^{+\infty} n
      w_{n} \left( z-z_{0} \right)^{n-1}
   \end{dmath*},
   \begin{dmath*}
      w''(z) = \rho (\rho -1) \left( z-z_{0} \right)^{\rho -2} \sum_{n=0}^{+\infty} w_{n} \left(
	 z -z_{0} \right)^{n} + 
      2 \rho \left( z-z_{0} \right)^{\rho -1} \sum_{n=0}^{+\infty} nw_{n} \left(
	 z-z_{0} \right)^{n-1} 
      + \left( z- z_{0} \right)^{\rho} \sum_{n=0}^{+\infty} n (n-1) 
      w_{n} \left( z-z_{0} \right)^{n-2}
   \end{dmath*}.
\end{dgroup*}
Moreover, by Cauchy formula for multiplying two absolutely convergent series%
\footnote{Namely, \begin{math}
      \left( \sum_{n=0}^{+\infty} a_{n} \right) 
      \left( \sum_{n'=0}^{+\infty} b_{n'}  \right) = \sum_{n=0}^{+\infty}
      c_{n}\end{math}, where \begin{math} c_{n} = \sum_{k=0}^{n} a_{k} b_{n-k}
   \end{math}.}
we get
\begin{dgroup*}
   \begin{dmath*}
      \sum_{n=0}^{+\infty} p_{n} \left( z-z_{0} \right)^{n} \sum_{n'=0}^{+\infty}
      w_{n'} \left( z-z_{0} \right)^{n'} 
      = \sum_{n=0}^{+\infty} \sum_{j=0}^{n}  w_{j} p_{n-j} \left( z-z_{0} \right)^{j+
	 n-j} 
      = \sum_{n=0}^{+\infty} \left[  \left( z-z_{0} \right)^{n} \sum_{j=0}^{n}  w_{j}
	 p_{n-j} \right] 
   \end{dmath*},
   \begin{dmath*}
      \sum_{n=0}^{+\infty} p_{n} \left( z-z_{0} \right)^{n} \sum_{n'=0}^{+\infty} n'
      w_{n'} \left( z-z_{0} \right)^{n'-1} 
      = \sum_{n=0}^{+\infty} \sum_{j=0}^{n}  w_{j} p_{n-j} \left( z-z_{0} \right)^{j+
	 n-j} 
      = \sum_{n=0}^{+\infty} \left[  \left( z-z_{0} \right)^{n} \sum_{j=0}^{n}  w_{j}
	 p_{n-j} \right] 
   \end{dmath*},
\end{dgroup*}
Plugging into Eq.~\eqref{eq:odeP},
\begin{dmath}
   \rho (\rho -1) \left( z-z_{0}\right)^{\rho} \sum_{n=0}^{+\infty} w_{n} \left(
      z-z_{0} \right)^{k} + 
   2\rho \left( z-z_{0} \right)^{\rho} \sum_{n=0}^{+\infty} n w_{n} \left( z-z_{0}
   \right)^{n} 
   + \left( z-z_{0} \right)^{\rho} \sum_{n=0}^{+\infty} n (n-1) w_{n} \left(
      z-z_{0} \right)^{n} +
   \rho \left( z-z_{0} \right)^{\rho} \sum_{n=0}^{+\infty} \left( z-z_{0}
   \right)^{n} \sum_{j=0}^{n} w_{j} p_{n-j} 
   + \left(z-z_{0}  \right)^{\rho} \sum_{n=0}^{+\infty} \left( z-z_{0} \right)^{n}
   \sum_{j=0}^{n} w_{j} j p_{n-j} + 
   \left( z-z_{0} \right)^{\rho} \sum_{n=0}^{+\infty} \left( z-z_{0} \right)^{n}
   \sum_{j=0}^{n} w_{j} q_{n-j} = 0
\end{dmath},
from which it follows 
\begin{dmath*}
   \sum_{n=0}^{+\infty} \left\lbrace 
   \left[ \rho (\rho-1)  + 2\rho n + n (n-1) \right] w_{n} +
   \sum_{j=0}^{n} \left[ (\rho + j) p_{n-j} + q_{n-j} \right] w_{j}  \right\rbrace
   \left( z-z_{0} \right)^{n+\rho} = 0
\end{dmath*}.
Notice that 
\begin{dmath*}
   \rho (\rho -1) + 2\rho n + n (n-1) = 
   \rho^{2} + \rho ( 2n -1) + n(n-1) = (\rho + n) (\rho + n -1) 
\end{dmath*}.
Therefore, 
\begin{dmath*}
   \sum_{n=0}^{+\infty} \left\lbrace 
   ( \rho + n ) (\rho + n -1) w_{n} +
   \sum_{j=0}^{n} \left[ (\rho + j) p_{n-j} + q_{n-j} \right] w_{j}  \right\rbrace
   \left( z-z_{0} \right)^{n+\rho} = 0
\end{dmath*}.
The coefficients must be zero, so 
\begin{dmath}
   ( \rho + n ) (\rho + n -1) w_{n} +
   \sum_{j=0}^{n} \left[ (\rho + j) p_{n-j} + q_{n-j} \right] w_{j}   =0 
\end{dmath}.
This is a system of infinitely many homogeneus linear coupled recursive
relations  involving one additional unknown parameter, which is $\rho$.

First of all, it is useful to simplify the notations. We set
\begin{dgroup*}
   \begin{dmath*}
      \lambda_{0} (\xi) = \xi (\xi-1) + p_{0} \xi + q_{0} 
      = \xi^{2} + \xi (p_{0} -1) + q_{0} 
   \end{dmath*}
   \begin{dsuspend}
      and
   \end{dsuspend}
   \begin{dmath*}
      \lambda_{n} (\xi) = p_{n} \xi + q_{n} \condition*{n\in\N\backslash\lbrace
	 0\rbrace}
   \end{dmath*}
\end{dgroup*}
for every $\xi\in\C$.
Then, for $n=0$ we get
\begin{dmath}[label={preindicial}]
   \lambda_{0} (\rho ) w_{0} = 0,
\end{dmath}
and for $n\in\N\backslash\lbrace0\rbrace$ we have
\begin{dmath*}
   ( \rho + n ) (\rho + n -1) w_{n} +
   \sum_{j=0}^{n} \left[ (\rho + j) p_{n-j} + q_{n-j} \right] w_{j} =
   ( \rho + n ) (\rho + n -1) w_{n} +
   \left[ (\rho + n) p_{0} + q_{0}  \right] w_{n} + 
   \sum_{j=0}^{n-1} \left[ (\rho + j) p_{n-j} + q_{n-j} \right] w_{j} =
   \lambda_{0} (\rho +n) w_{n} + \sum_{j=0}^{n-1} \lambda_{n-j} (\rho + j) w_{j}
\end{dmath*},
so
\begin{dmath}
   \lambda_{0} (\rho +n) w_{n} + \sum_{j=0}^{n-1} \lambda_{n-j} (\rho + j) w_{j}=0
\end{dmath}.
Since $w_{0} \neq 0$, 
Eq.~\eqref{eq:preindicial} yields
\begin{dmath}[label={indicial}]
   \lambda_{0}(\rho) = 0
\end{dmath},
or more explicitly
\begin{dmath*}
   \rho^{2} + (p_{0} -1) \rho + q_{0} =0 
\end{dmath*}.
Eq.~\eqref{eq:indicial} \index{Indicial equation} is called ``indicial equation'' associated with the
point $z_{0}$ and the differential equation~\eqref{eq:ode}.
%\end{proof}

\section{The point at infinity}
\label{sec:point at infinity}

\section{Totally Fuchsian equations}

\begin{approfondimento}
   Integrals of Fuchsian differential equations are a source of
   trascendental functions.
   \begin{flushright}
      --- Folklore
   \end{flushright}
\end{approfondimento}

\begin{definition}
   Eq.~\eqref{eq:ode} is called ``of Fuchsian class'' or ``totally
   Fuchsian'' if all its singular points  in the \emph{extended} complex plane are
   Fuchsian singular points.
\end{definition}

The definition includes the point at infinity. 
So far, the isolated singularities of eq.~\eqref{eq:ode} has been
assumed to be finite  in number \emph{by hypothesis}, it is clear that a
totally Fuchsian equation \emph{must} have at most finitely many singular point.


\section{Interlude on M\"obius transformations}

\subsection{Definition and first properties}

In this section, we shall study mappings of the extended complex plane onto
itself of the form
\begin{dmath}[label={moebius}]
   w (z) = \frac{ a z + b}{cz +d } ,
\end{dmath}
where $a$, $b$, $c$, $d$ are arbitrary complex numbers which satisfy
\begin{dmath}
   ad -bc \neq 0.
\end{dmath}
Since multiplying all these numbers by a common non-zero number does not alter
the mapping,  it is always possible to assume that 
\begin{dmath}
   ad -bc = 1 .
\end{dmath}
Eq.~\eqref{eq:moebius} is referred to as ``M\"obius transformation'', in honor
of  the mathematician M\"obius 
(the same of the famous
strip).
It is also sometimes called ``homographic transformation'' or in short ``homography''.

Due to the fact that these transformations  play an important role in various fields of pure and applied Mathematics and
Physics, it seems useful to study their properties more carefully than what is
needed for the purposes of the following sections.

%A nice book to study these topics is \textcite{Blair:2000}. 
%Most of this section is based from that book, and some other material is taken
%from 
%\textcite{Hille:1959,Ablowitz.Fokas:2003}. 

\begin{theorem}
   Any M\"obius transformation which is not  a linear function can always be obtained as the composition of two
   linear transformations and one inversion.
\end{theorem}
\begin{proof}
   If $c=0$ we get a linear transformation:
   \begin{dmath*}
      w(z) = \frac{a}{d} z + \frac{b}{d} .
   \end{dmath*}
   In the case $c\neq 0$, one can notice that 
   \begin{dmath*}
      \frac{az+b}{cz+d} = \frac{a}{c} + \frac{bc -ad}{c( cz +d)}, 
   \end{dmath*}
   which shows explicitly that the transformation may be obtained equivalently
   through the composition of 
   \begin{dgroup*}
      \begin{dmath*}
	 w_{1}(z) = cz +d ,
      \end{dmath*}
      \begin{dmath*}
	 w_{2} (w_{1}) = \frac{1}{w_{1}},
      \end{dmath*}
      \begin{dmath*}
	 w_{3} (w_{2}) = \frac{a}{c} + \frac{bc-ad}{c} w_{2} ,
      \end{dmath*}
   \end{dgroup*}
   so the theorem is proven.
\end{proof}

\subsection{Group structure}

\begin{theorem}
   The set of all and only M\"obius transformations on the extended complex plane
   form a group.
\end{theorem}


\begin{proof}
   Consider the set of all and only the M\"obius transformations on the extended
   complex $z$-plane.
   The composition of any two M\"obius functions is again a M\"obius function. 
   We have to prove that 
   \begin{aenumerate}
   \item
      Composition of M\"obius transformations is associative;
   \item
      Among all and only the M\"obius functions, there exists an identity
      transformation, \ie, a M\"obius transformation such that 
   \item
      There exists a null element, \ie, a M\"obius transformation which, after
      composition  with
      the identity, remains the same.
   \item
      Any M\"obius transformation is invertible. 
   \end{aenumerate}
\end{proof}

\section{The Papperitz-Riemann equation}

The main result of this section is
\begin{approfondimento}
   All second-order totally Fuchsian equations having precisely
   three regular singular points in the extended complex plane
   are \emph{uniquely} determined by the position of their
   singular points and the corresponding pairs of characteristic exponents at
   these points.
\end{approfondimento}

\begin{dmath*}
   \PRiemann{a,\alpha,\alpha'}{b,\beta,\beta'}{c,\gamma,\gamma'}{z}.
\end{dmath*}
