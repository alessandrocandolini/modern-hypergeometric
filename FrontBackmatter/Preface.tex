\chapter*{Preface}
\markboth{\spacedlowsmallcaps{Preface}}{\spacedlowsmallcaps{Preface}}
%\addtocontents{toc}{\protect\vspace{\beforebibskip}} % to have the bib a bit from the rest in the toc
\addcontentsline{toc}{chapter}{\tocEntry{Preface}}

Hypergeometric functions  are useful in many different branches of applied
mathematics, theoretical physics, statistics, etc.  Part of the  explanation for
this fact is that (\S~\ref{chap:fuchs}) integrals of any
second-order linear homogeneus ordinary differential equation of Fuchsian class
having precisely three given regular singular points in the extended complex
plane can ultimately be expressed in closed form using hypergeometric
functions (\S~\ref{chap:hyper}). Among them are Legendre
functions, Jacobi polynomials, Chebychev polynomials, and more.  Other differential
equations which are not of this form can happen to 
be confluent forms of the hypergeometric equation and in that case
they can be integrated in finite terms by means of confluent hypergeometric
functions (\S~\ref{chap:confluent}). Bessel equation belongs to this class of
equations. The approach in these notes is conventional, eg, using the so-called ``multi-valued'' functions instead of relying on a more rigorous and elegant approach based on Riemann surfaces.

\smallskip

\noindent\textsw{\myLocation, \myTime}


\begin{flushright}
        \myName
\end{flushright}

